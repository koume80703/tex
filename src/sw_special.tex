\documentclass{jsarticle}
\title{ソフトウェア基礎特論 期末レポート}
\author{4年15組28番 高野昂平}
\date{}

\usepackage{graphicx}

\begin{document}
\maketitle
\section{ゲームとプレイヤー}
ゲームとしては、\textgt{オセロ}を選んだ。先手のプレイヤーを黒とし、モンテカルロ木探索を用いてその強さを評価する。また、後手はランダムの場合と、MCTSを実装した場合の2パターンを試した。\par
モンテカルロ木探索の子ノード選択の基準としてはUCB1を選択した。以下の式の$w$はそのノードにおける評価値、$n$はそのノードの訪問回数、$N$はそれぞれの子ノードの訪問回数の合計値、$C$は定数で探索がうまくいくようなパラメータとなっている。第2項が訪問回数が少ないほど大きくなる項であり、訪問回数が少ないノードに重きを置く場合、定数$C$の値を大きくすべきである。ただし、大きくしすぎると、第1項の評価値が軽視されるため、調整が必要となっている。今回の実装では、$C=\sqrt{2}$とした。
\begin{eqnarray}
    \frac{w}{n} + C\sqrt{\frac{\ln{N}}{n}} \nonumber
\end{eqnarray}
\section{実装と実験環境}
実装にあたっては、言語をPythonとし、MacOS上での実験を行った。
\section{勝率について}
各条件のもと、20戦の対戦を行い、その勝率を評価する。本来は100戦ほど対戦を行わせるべきではあるが、実行時間との兼ね合いを見て、20戦とした。\par
MCTSプレイヤーにおけるパラメータは、1手を導くのに何回の評価を行うのか(evaluate number)と、ノードを展開する際に何回訪問したノードを展開するか(expand base number)、の2つである。以下の表において、evaluate numberをeval\_num、expand base numberをeb\_numとしている。
\newpage
\subsection{対ランダムプレイヤー}
表\ref{tab:vsRandom}にランダムプレイヤーとの対戦の結果を示す。\par
\begin{table}[hbtp]
    \caption{対ランダムプレイヤーでの勝率}
    \label{tab:vsRandom}
    \begin{center}
        \scalebox{1.3}{
            \begin{tabular}{cc||ccc}
                \multicolumn{2}{c}{パラメータ} & \multicolumn{3}{c}{勝率} \\ \hline
                eval\_num & eb\_num & 勝ち & 負け & 引き分け \\ \hline
                1 & 0 & 12 & 8 & 0 \\ \hline
                5 & 2 & 12 & 7 & 1 \\ \hline
                10 & 3 & 15 & 5 & 0 \\ \hline
                20 & 5 & 14 & 4 & 2 \\ \hline
                30 & 10 & 19 & 0 & 1 \\ \hline
                50 & 10 & 20 & 0 & 0 \\ \hline
                100 & 30 & 20 & 0 & 0 \\ \hline
                150 & 50 & 19 & 1 & 0 \\ \hline
                200 & 70 & 20 & 0 & 0 
            \end{tabular}
        }
    \end{center}
\end{table}
結果を見てわかるように、MCTSプレイヤーとランダムプレイヤーの対戦においてはMCTSプレイヤーが勝ち越していることがわかる。ただし、評価回数(eval\_num)が1回であったり、5回であったりするものに関しては、ほとんどMCTSによる評価が行われておらず、この勝率が正しいものであるかは不明である。\par
その他のMCTSプレイヤーについては、特に、評価回数が30回を超えるものに関しては表を見てわかるように勝率が9割を超えていることから、ランダムプレイヤーに対して圧倒的な強さを持っていることがわかる。
\newpage
\subsection{対MCTSプレイヤー}
以下の表\ref{tab:vsMCTS}にMCTSプレイヤー同士の対戦結果を示す。\par
\begin{table}[hbtp]
    \caption{対MCTSプレイヤーでの勝率}
    \label{tab:vsMCTS}
    \begin{center}
        \scalebox{1.3}{
            \begin{tabular}{cc|cc||ccc}
                \multicolumn{4}{c}{パラメータ} & \multicolumn{3}{c}{勝率} \\ \hline
                \multicolumn{2}{c}{先手(黒)} & \multicolumn{2}{c}{後手(白)} & & & \\ \hline
                eval\_num & eb\_num & eval\_num & eb\_num & 勝ち & 負け & 引き分け \\ \hline
                100 & 30 & 100 & 50 & 13 & 7 & 0 \\ \hline
                200 & 50 & 100 & 50 & 14 & 5 & 1 \\ \hline
                200 & 50 & 300 & 50 & 9 & 11 & 0 \\ \hline
                200 & 70 & 1000 & 200 & 6 & 14 & 0 
            \end{tabular}
        }
    \end{center}
\end{table}
パラメータの異なるプレイヤー同士で戦わせることでどのパラメータがMCTSプレイヤーの強さに影響を与えるか知ることができる。そして、この表を見てわかるのが、まず、表の1行目を見ると、評価回数が同じ場合、eb\_numの値が少ない方が強いということである。ただ、対戦回数が少ないことから、たまたま、eb\_numの値が少ない方が強いということである。これは、このeb\_numの値が大きすぎる場合、有望な手を深く読むことができず、正しく評価できないのではないかと思う。\par
次に、評価回数が多い方が強いということが表の2行目、3行目、4行目からわかる。やはり、評価回数の多い方がより正確に次の強い手を導くことができるということだと思う。
\end{document}