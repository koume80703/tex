\documentclass{jsarticle}
\title{人工知能特論 期末レポート}
\author{M1-55-17 高野 昂平}
\date{}

\begin{document}
\maketitle

\section*{設問2 最短経路}
\begin{itemize}
\item $V(m)$ : ノード$m$からノード7(終点)までの最短距離
\item $T_{mn}$ : 隣接した2つのノード$m$,ノード$n$間の距離
\end{itemize}
\par
$V(m) = \min [T_{mn} + V(n)]$を用いて再帰的に最短経路を求めていく。全体の最短経路を求めるには$V(1)$を求めれば良いので
\begin{eqnarray}
    V(1) = \min [T_{1n} + V(n)] = \min \{T_{12} + V(2), T_{13} + V(3)\} = \min \{3 + V(2), 2 + V(3)\} \nonumber 
\end{eqnarray}
この式の中で$V(2)$は以下のようにして同様に求めることができる。
\begin{eqnarray}
    V(2) = \min [T_{2n} + V(n)] = \min \{T_{24} + V(4), T_{25} + V(5)\} = \min \{4 + V(4), 2 + V(5)\} \nonumber
\end{eqnarray}
この手順を繰り返していき、$V(3)~V(6)$を求めていく。なお、$V(7)$はノード7からノード7への自身への距離を意味するのでその値は0とする$(V(7) = 0)$。
\begin{eqnarray}
    V(3) &=& \min [T_{3n} + V(n)] = \min \{T_{35} + V(5)\} = \min \{4 + V(5)\} \nonumber \\
    V(4) &=& \min [T_{4n} + V(n)] = \min \{T_{45} + V(5), T_{47} + V(7)\} = \min \{1 + V(5), 6 + V(7)\} = \min \{1 + V(5), 6\}\nonumber \\
    V(5) &=& \min [T_{5n} + V(n)] = \min \{T_{56} + V(6), T_{57} + V(7)\} = \min \{2 + V(6), 7 + V(7)\} = \min \{2 + V(6), 7\}\nonumber \\
    V(6) &=& \min [T_{6n} + V(n)] = \min \{T_{67} + V(7)\} = \min \{4 + V(7)\} = 4\nonumber
\end{eqnarray}
ここで、$V(6) = 4$を元に代入を順に行なっていくと、
\begin{eqnarray}
    V(5) &=& \min \{2 + V(6), 7\} = \min \{2 + 4, 7\} = 6 \nonumber \\
    V(4) &=& \min \{1 + V(5), 6\} = \min \{1 + 6, 6\} = 6 \nonumber \\
    V(3) &=& \min \{4 +V(5)\} = \min \{4 + 6\} = 10 \nonumber \\
    V(2) &=& \min \{4 + V(4), 2 + V(5)\} = \min \{4 + 6, 2 + 6\} = 8 \nonumber \\
    V(1) &=& \min \{3 + V(2), 2 + V(3)\} = \min \{3 + 8, 2 + 10\} = 11 \nonumber
\end{eqnarray}
よって、求める最短経路は$1 \rightarrow 2 \rightarrow 5 \rightarrow 6 \rightarrow 7$で距離は11となる。
\end{document}