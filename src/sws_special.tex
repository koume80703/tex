\documentclass{jsarticle}
\title{ソフトウェア科学特論 期末レポート}
\author{M1-55-17 高野 昂平}
\date{}

\usepackage{url}
\usepackage[dvipdfmx]{hyperref}
\usepackage{pxjahyper}
\hypersetup{
    hidelinks=true,
    bookmarks=false,
}

\begin{document}
\maketitle

\section{テーマについて}
スケーラブルなシステムとしてMicrosoft Azureについて取り上げる。
\subsection{Microsoft Azureとは}
Microsoft Azure(以下、Azure)とは、IaaS(Infrastructure as a Service)とPaaS(Plattform as a service)を提供するクラウドサービスである。実際に、ユーザーがクラスタを持つ必要はなく、メンテナンスなどをAzure側に任せることにより、ユーザー側はその他の業務にリソースを割くことができるため、効率よく業務を行うことができる。
\par Azureは様々なツールやフレームワークを提供しており、データ分析や開発業務に役立てることができるサービスとなっている。Azureを用いたユーザーとしては、行政や医療、製造など幅広い業種でAzureが用いられている。
\section{手法について}
\subsection{データセンター}
Azureは、他のクラウドサービス(AWSなど)と同様に物理的なデータセンターを持つ。1つ以上のデータセンターの集合のことをリージョンと呼ぶが、Azureでは東日本リージョンと西日本リージョンの2つのリージョンがある。東日本リージョンでは東京、埼玉に、西日本リージョンでは大阪にデータセンターを持つ。
\subsection{仮想化技術}
ほとんどのハードウェア資源を仮想ハードウェアとして利用でき、ソフトウェアでの命令をハードウェアの命令に置き換えることが可能となっている。
\par ハードウェア資源は先述のデータセンターにあるサーバー上のもので、各サーバーではVMM(仮想マシンモニター、仮想マシンマネージャー)を用いて、資源を管理している。それにより、実際のハードウェアを仮想化し、ソフトウェア上でその仮想化されたハードウェアを使用して実行することが可能となった。
\subsection{サーバー}
複数のサーバーをサーバーラックに配置し、1つの単位としている。また、そのラックを複数まとめてクラスターという大きなグループへ分類することもある。
\par それぞれのサーバラックにはブレードサーバとネットワークスイッチが搭載されており、これらはネットワークの接続と電力の供給の役割を担っている。
\subsection{各ユーザーへのインスタンスの生成}
各ユーザーはクラウドのサービスを受ける以上、ハードウェアを割り当てられる必要がある。そのため、その割り当てをインスタンスとしてAzureのシステム上で生成する必要がある。各インスタンスは、サーバラックまたはクラスターを選択し、その上でユーザーは業務を行うことになる。
\par Azureでは、提供したサーバーラックの一部でファブリックコントローラーと呼ばれるクラウド管理システムを動作させている。これにより、サーバーへのサービスの割り当てやサーバーとサービスの状態を監視を行い、監視する中でエラーを発見した場合、そのエラーからサーバーを回復させることが可能となっている。
\section{スケーラビリティについて}
Azureはオートスケールという機能を提供しており、よく利用されているアプリ、サービス、利用されていないリソースをシステムが判断し、柔軟にリソースの拡張、縮小を行うことができる。これにより、不要なリソースを別のサービスやアプリに割り当てることができ、ユーザー側としてもリソースの監視を行う必要がなく、別の業務に専念することができる。
\par 信頼性の面では、Azureのデータセンターが先述の日本の2リージョンの他、全世界に160箇所のデータセンターを持っており、これにより使用しているデータセンターにて災害による障害が発生したとしても、別のデータセンターにてユーザーのサービスを復旧させ、再開することが可能となっている。
\section{実験の考察}
Azure上で、アプリケーションを実行し、そのアプリケーションによる負荷を徐々に大きくしていく。大きくしていく限界として、そのアプリケーションがAzure上の環境上で明らかにストレスがかかっているとわかるまで行う。その後、さらに負荷を大きくした同じアプリケーションを実行するが、その際の環境は最初の環境よりもリソースの多い環境で実行する。このリソースが増えた環境はAzure側でリソースを増やした環境を用意し、それを利用する形になる。この新しいリソースを増やした環境でも、アプリケーションが動くのであれば、以降再び負荷が限界になった時に、リソースを拡張することでまたアプリケーションが問題なく動くことが予想されるため、Azureがスケーラビリティを持つサービスであると言える。
\begin{thebibliography}{9}
    \bibitem{Web:Azure1} \href{https://www.jbcc.co.jp/blog/column/azure-mechanism.html}{Microsoft Azure(アジュール)とは何が出来る?わかりやすく解説【初心者向け】}
    \bibitem{Web:Azure2} \href{https://learn.microsoft.com/ja-jp/azure/cloud-adoption-framework/get-started/what-is-azure}{Azure のしくみ}
    \bibitem{Web:Azure3} \href{https://www.fsi.co.jp/blog/5520/}{【第1回】Microsoft Azureとは~Azureの基本性能と全体像を見る~}
\end{thebibliography}
\end{document}