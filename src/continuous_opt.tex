\documentclass{jsarticle}
\title{連続最適化特論 期末レポート}
\author{M1-55-17 高野昂平}
\date{}

\usepackage{bm}
\usepackage{listings, jvlisting}
\lstset{
    language = {Python},
    frame = single,
    breaklines = true,
    numbers = left
}

\begin{document}
\maketitle
\section*{課題について}
課題内容として、不動点近似法の3つのアルゴリズム(POCS, Krasnosel'skii-Mann, Halpern)について挙動を調べた。
\par 言語はPythonを用いた。
\section*{POCS}
POCSアルゴリズムを実装する上で、閉球を用いるため、以下のような簡単な閉球クラスを作成した。
\begin{lstlisting}[caption = 閉球クラス]
    class ClosedBall:
        def __init__(self, center, radius):
            self.center = np.array(center)
            self.radius = radius

        def projection(self, coordinate):
            norm = np.linalg.norm(coordinate - self.center)
            if norm <= self.radius:
                return coordinate
            else:
                return self.center + (self.radius / norm) * (coordinate - self.center)
\end{lstlisting}
\par この閉球クラスは中心centerと半径radiusを元に作成されるクラスで、射影を行うメソッドprojectionを持つ。
\large
\begin{eqnarray}
    P(\bm{x}) = \left\{
        \begin{array}{ll}
            \bm{c} + \frac{r}{\|\bm{x} - \bm{c}\|}(\bm{x} - \bm{c}) & (\bm{x} \notin B) \\
            \bm{x} & (\bm{x} \in B)
        \end{array}
    \right.
\end{eqnarray}
\normalsize
projectionメソッドはこの射影$P(\bm{x})$に相当し、projectionメソッドの取る引数coordinateは$P(\bm{x})$の$\bm{x}$にあたる。
\par 挙動を調べるにあたって、2次元平面上の4つの円における不動点近似を例に見ていく。
\begin{lstlisting}[caption = 4つの円における不動点近似]
    def main():
        cb1 = ClosedBall((3, 3), 5)
        cb2 = ClosedBall((0, 0), 3)
        cb3 = ClosedBall((-3, -3), 10)
        cb4 = ClosedBall((2, 1), 7)

        coordinate = cb4.projection(np.array((21, 5)))
        print(coordinate)
        coordinate = cb3.projection(coordinate)
        print(coordinate)
        coordinate = cb2.projection(coordinate)
        print(coordinate)
        coordinate = cb1.projection(coordinate)
        print(coordinate)
\end{lstlisting}
\begin{lstlisting}[caption = 出力結果]
    [8.84984849 2.44207337]
    [6.08748157 1.17344926]
    [2.9457696  0.56783928]
    [2.9457696  0.56783928]
\end{lstlisting}
\par 4つの円として、cb1,cb2,cb3,cb4を生成した(Listing1 2-5行目)。この生成した円はそれぞれprojectionメソッドを用いて、射影を返すことができ、その射影をそれぞれ$P_1, P_2, P_3, P_4$とすると、不動点$T$は$T = P_1P_2P_3P_4(\bm{x})$となる。今回は挙動を調べるため、$P_4(\bm{x}), P_3P_4(\bm{x}) , P_2P_3P_4(\bm{x}), P_1P_2P_3P_4(\bm{x})$を順に出力している(Listing2 7-14行目)。$\bm{x} = (21, 5)$とした。
\par $P_4(\bm{x}), P_3P_4(\bm{x}) , P_2P_3P_4(\bm{x})$の計算においては、与えられた座標が円外にあるため、計算が行われるが、$P_1P_2P_3P_4(\bm{x})$の計算においては与えられた座標がcb1の円内にあるため、与えられた座標がそのままcb1の射影となっている(式(1))。
\newpage
\section*{Krasnosel'skii-Mann}
POCSと同様に4つの円を生成し、どのように不動点近似が行われていくのかその挙動を見ていく。
\par まず、非拡大写像として以下のようなメソッドを実装した。
\begin{lstlisting}[caption = 非拡大写像]
    def non_expansion_map(cb_list, coordinate):
        sum = np.array((0, 0))
        for cb in cb_list[1:]:
            sum = sum + cb.projection(coordinate)
        else:
            average = sum / (len(cb_list) - 1)
        return cb_list[0].projection(average)
\end{lstlisting}
\par このメソッドは以下の式
\large
\begin{eqnarray}
    T(\bm{x}) = P_0\left(\frac{1}{m}\sum^m_{i=1}P_i(\bm{x})\right) \nonumber
\end{eqnarray}
\normalsize
をプログラム上に落とした式である。
\par 次に、再帰的に不動点近似を行うメソッドkmを以下に示す。
\begin{lstlisting}[caption = Krasnosel'skii-Mann メソッド]
    def km(coordinate, cb_list, alpha = 0.5):
        new_coordinate = alpha * coordinate + (1 - alpha) * non_expansion_map(
            cb_list, coordinate
        )
        norm = np.linalg.norm(new_coordinate - coordinate)
        print(new_coordinate, norm)
        if norm < 10**-6:
            return new_coordinate
        else:
            return km(new_coordinate, cb_list)
\end{lstlisting}
このメソッドは、以下の式をプログラム上に落としたもので、終了条件を$||T(\bm{x}) - \bm{x}|| < 10^{-6}$としている。また、ステップ幅$\alpha$はデフォルトで0.5としている。
\large
\begin{eqnarray}
    \bm{x}_{n+1} = \alpha\bm{x}_n - (1 - \alpha)T(\bm{x}_n) \nonumber
\end{eqnarray}
\normalsize
実行時のmain文は以下の通りで、POCSの実験の時と同様の円を4つ生成している。また、初期点も同様のものとした。
\begin{lstlisting}[caption = main文]
    def main():
        cb0 = ClosedBall((3, 3), 5)
        cb1 = ClosedBall((0, 0), 3)
        cb2 = ClosedBall((-3, -3), 10)
        cb3 = ClosedBall((2, 1), 7)
        cb_list = [cb0, cb1, cb2, cb3]
        km(np.array((21, 5)), cb_list)
\end{lstlisting}
\begin{lstlisting}[caption = 出力結果]
    [13.54251669  3.04986877] 7.708246829925411
    [9.80634186 2.07789922] 3.8605345646038773
    [7.93066159 1.58316718] 1.939828931025051
    [6.81445695 1.29779227] 1.1521074738158075
    [6.06084541 1.12728637] 0.7726594470796265

                    ........

    [2.94942643 0.54857829] 1.8142098167533117e-06
    [2.94942495 0.54857802] 1.511841513954326e-06
    [2.94942371 0.54857779] 1.259867928364655e-06
    [2.94942268 0.54857759] 1.0498899403072627e-06
    [2.94942182 0.54857743] 8.74908283890604e-07
\end{lstlisting}
\par 出力行数が非常に長くなるため、一部割愛したが、実行結果としては79回の再帰にて終了条件を満たし、不動点近似を完了した。POCSでの近似と近い結果となっていることがわかる。
\section*{Halpern}
このアルゴリズムにおいても、Krasnosel'skii-Mannアルゴリズムと同様の非拡大写像を用いた(Listing4)。以下がHalpernアルゴリズムを実装したものである。
\begin{lstlisting}[caption = Halpern メソッド]
    def halpern(coordinate, cb_list, step, initial):
        alpha = 1 / (2**step)
        new_coordinate = alpha * initial + (1 - alpha) * non_expansion_map(
            cb_list, coordinate
        )
        next_step = step + 1
        norm = np.linalg.norm(new_coordinate - coordinate)
        print(new_coordinate, norm)
        if norm < 10**-6:
            return new_coordinate
        else:
            return halpern(new_coordinate, cb_list, next_step, initial)
\end{lstlisting}
\par 1-4行目は以下の式をプログラム上に起こしたものである。
\begin{eqnarray}
    \alpha_k &=& \frac{1}{2^k} \nonumber \\
    \bm{x}_{k+1} &=& \alpha_k\bm{x}_0 + (1 - \alpha_k)T(\bm{x}_k) \nonumber
\end{eqnarray}
\par Halpernの挙動確認においても同様に4つの円と初期点を元に不動点近似を行った。円の実装についてはPOCSにて使用したClosedBallクラスを用いている。
\begin{lstlisting}[caption = main文]
    def main():
        cb0 = ClosedBall((3, 3), 5)
        cb1 = ClosedBall((0, 0), 3)
        cb2 = ClosedBall((-3, -3), 10)
        cb3 = ClosedBall((2, 1), 7)
        cb_list = [cb0, cb1, cb2, cb3]
        coordinate = np.array((21, 5))
        halpern(coordinate, cb_list, 1, coordinate)
\end{lstlisting}
\begin{lstlisting}[caption = 出力結果]
    [13.54251669  3.04986877] 7.708246829925411
    [9.80262528 2.07944725] 3.8637424430288365
    [7.92277551 1.57865309] 1.9454125408529224
    [6.65232618 1.25963155] 1.3098916863739631
    [5.74176409 1.07823057] 0.9284555086952968
    
                    ........

    [2.94470308 0.57338062] 3.4641723959355285e-06
    [2.94470081 0.57338018] 2.3094820396481456e-06
    [2.9446993  0.57337988] 1.5396715812279111e-06
    [2.9446983  0.57337969] 1.0264561652676736e-06
    [2.94469762 0.57337956] 6.843083314337784e-07
\end{lstlisting}
\par こちらも同様に、出力結果が長いので一部割愛した。実行結果としては42回の再帰にて終了条件を満たした。ステップ幅がKrasnosel'skii-Mannアルゴリズムと異なり、最初は大きく不動点に近づくため、再帰回数が少なくなっていると考えられる。
\end{document}
